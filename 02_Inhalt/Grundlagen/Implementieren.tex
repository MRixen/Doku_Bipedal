\section{Installation}
\label{sec:Installation}
In diesem Kapitel wird die Vorgehensweise erläutert, um die \gls{app} Fast 
Diagnose mit einer ABB-Steuerung nutzen zu können. Für das Einspielen der 
entsprechenden Dateien wird Robotstudio 5.61 verwendet.\\
Das gesamte ABB-System ist unter folgendem Link verfügbar: \\
\url{https://github.com/MRixen/Abb-Controller/}.\\
Das gesamte Android-System ist unter folgendem Link verfügbar: \\
\url{https://github.com/MRixen/rbc_Diagnose/}.\\
Die Dokumentation ist unter folgendem Link verfügbar: \\
\url{https://github.com/MRixen/Doku_App/}

\subsubsection{Vorbereitung}
Folgende Komponenten müssen in das bestehende ABB-Projekt 
integriert werden:
\begin{itemize}
\item Task ServerComm mit Modul ServerComm (s. \ref{fig:taskServerComm})
\begin{figure}[H]
	\centering
	\fbox{	
	\includegraphics[width=0.6\textwidth]{03_Grafiken/taskServerComm.jpg}}
	\caption[Task und Modul ServerComm]{Task und Modul ServerComm}
	\label{fig:taskServerComm}
\end{figure}
\item Task und Modul EventMessages (s. \ref{fig:taskEventMessages})
\begin{figure}[H]
	\centering
	\fbox{	
	\includegraphics[width=0.6\textwidth]{03_Grafiken/taskEventMessages.jpg}}
	\caption[Task und Modul EventMessages]{Task und Modul EventMessages}
	\label{fig:taskEventMessages}
	\end{figure}
\item Task und Modul MachineData (s. \ref{fig:taskMachineData})
	\begin{figure}[H]
		\centering
		\fbox{	
		\includegraphics[width=0.6\textwidth]{03_Grafiken/taskMachineData.jpg}}
		\caption[Task und Modul MachineData]{Task und Modul MachineData}
		\label{fig:taskMachineData}
		\end{figure}
\item Task und Modul CycleTimer (s. \ref{fig:taskCycleTimer})
\begin{figure}[H]
	\centering
	\fbox{	
	\includegraphics[width=0.6\textwidth]{03_Grafiken/taskCycleTimer.jpg}}
	\caption[Task und Modul CycleTimer]{Task und Modul CycleTimer}
	\label{fig:taskCycleTimer}
\end{figure}

\end{itemize}  
Folgende Komponenten müssen in dem bestehenden ABB-Projekt 
ergänzt/modifiziert werden:
\begin{itemize}
\item Automatik loading of Modules (s. \ref{fig:autoLoad})
\begin{figure}[H]
	\centering
	\fbox{	
	\includegraphics[width=0.6\textwidth]{03_Grafiken/autoLoad.jpg}}
	\caption[Automatic loading]{Automatic loading}
	\label{fig:autoLoad}
\end{figure}
\item I/Os (s. \ref{fig:signale})
\begin{figure}[H]
	\centering
	\fbox{	
	\includegraphics[width=0.6\textwidth]{03_Grafiken/signals.jpg}}
	\caption[Signale hinzufügen]{Signale hinzufügen}
	\label{fig:signale}
	\end{figure}
\item Modifikation von Global.sys
\begin{figure}[H]
	\centering
	\fbox{	
	\includegraphics[width=1\textwidth]{03_Grafiken/globalSys.jpg}}
	\caption[Global.sys]{Global.sys}
	\label{fig:globalSys}
\end{figure}
Die Buffer, sowie die Variable \textit{clientConnected} müssen global für jede 
Task verfügbar sein.
\end{itemize}  

\subsubsection{Verwendung}
Sobald alle Tasks aktiv sind, muss an dem Netzwerkanschluss der 
Robotersteuerung ein Wlan-Router angebunden werden. Die bei ABB einzig 
verwendbare IP-Adresse lautet: \textbf{192.168.1.2} und der Port ist 
standardmäßig auf \textbf{1025} gesetzt.\\
Mit Hilfe eines QR-Codes, der die Adressdaten in folgendem Format enthält:\\ 
\textbf{192.168.1.2;1025} (s. \ref{fig:qrCode}) kann sich nun auf den Port 
verbunden werden.
\begin{figure}[H]
	\centering
	\fbox{	
	\includegraphics[width=0.4\textwidth]{03_Grafiken/qrCode.png}}
	\caption[QR-Code mit 192.168.1.2;1025]{QR-Code mit 192.168.1.2;1025}
	\label{fig:qrCode}
\end{figure}
