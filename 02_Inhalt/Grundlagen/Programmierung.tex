\subsection{Programmierung}
\subsubsection{OpenCM Code}
Das OpenCM Board wird mit der gls{ide} Robotis OpenCM (modifizierte 
Arduino-Umgebung) programmiert, da das Programm sehr schlicht gehalten wird.
Der Speicherplatz von OpenCM ist stark begrenzt, sodass dieses nur als 
Befehlsverteiler fungiert.\\
Das bedeutet, dass ein übergeorndneter Rechner (geplant ist Raspberry Pi) eine 
in C\# geschriebene \gls{hmi} enthält mit der die komplette Regelung aktiviert 
und (in gewissen Mae) modifiziert werden kann.\\
Das Board OpenCM empfängt die vom Rasbperry Pi gesendeten Befehle und leitet 
diese an die entsprechenden Aktoren weiter.

Das Programm, welches auf dem OpenCM-Controller läuft ist in 
\ref{lst:OpenCMcode} dargestellt.\\
In Zeile 1-3 erfolgt das Definieren des seriellen Bus, welche verwendet werden 
können. Da das Extension Board zusammen mit dem OpenCM Controller Verwendung 
findet wird in Zeile 9 der Bus 3 übergeben.\\
Jeder Aktor erhält eine ID (zwischen 0 und 253) und ist aktuell (wie in Zeile 4 
zu sehen) auf 1 gesetzt. 
Die Befehle, welche an den Servomoter geschickt werden können, um Daten zu 
setzen oder abzurufen sind in \ref{tab:CommandsDynamixel} dargestellt. In Zeile 
6 erfolgt die Definition des Befehls 30, die den Aktor zu einer Position (0-300 
$[$deg$]$, bzw. 0-1024).\\
In der setup-Routine (Zeile 11-15) erfolgt das Setzen der Baudrate (s. 
\ref{tab:baudrate}) und des Bewegugnsmodus (Joint oder Wheel) für den 
entsprechenden Aktor. 
Sobald Daten auf der seriellen Schnittstelle vorliegen werden diese in Zeile 
19-23 mit der Funktion getValue() abgerufen und an die Funktion Dxl.writeWord() 
(s. \ref{lst:dxlWriteWord} übergeben.
\label{lst:dxlWriteWord}
\begin{lstlisting}[language=java, caption=Funktion dxlWriteWord]
void dxl_write_word(
   int id,
   int address,
   int value
);
\end{lstlisting}

Die Funktion getValue() liest die Anzahl der gesendeten Zeichen und splittet 
diese auf in die ID, den Befehl und den Wert für den Befehl (Zeile 17-XX).

\label{lst:OpenCMcode}
\begin{lstlisting}[language=java, caption=OpenCM-Programm]
#define DXL_BUS_SERIAL3 3
#define ID_NUM 1
#define DIM_ID_ARRAY 3
#define DIM_CMD_ARRAY 2
#define DIM_VAL_ARRAY 4

Dynamixel Dxl(DXL_BUS_SERIAL3);
  int id;
  int command;
  int sollPosition;

void setup() {
  Dxl.begin(3);
  Dxl.jointMode(ID_NUM);
}

void loop() {
   id = getValue();
   command = getValue();
   sollPosition = getValue();
   if((id != -1) & (command != -1) & (sollPosition != -1)) Dxl.writeWord(id, 
   command, sollPosition);
   delay(100);
}

int getValue(){
  int inValue;
  int cntr;
  int motorId[DIM_ID_ARRAY];
  int command[DIM_CMD_ARRAY];
  int value[DIM_VAL_ARRAY];
   
  // Check if there is something to receive
  cntr = SerialUSB.available();  
  if(cntr>0){
    // Read the first parameter to the first semicolon
  for(unsigned int i=0; i < cntr;i++){
      inValue = (SerialUSB.read())-48;
      if(inValue != 11) {
        motorId[i] = inValue; 
      }
      else return calcValue(motorId, i,calcMult(i));;
    }
    }
    else return -1;
  } 

// Caclulate the multiplicator
int calcMult(int steps){
    int multiplicator = 1;
    
    for(unsigned int i=0; i < steps-1;i++){
      multiplicator = multiplicator*10;
   }
   return multiplicator;
}

// Calculate the value
int calcValue(int data[], int dim, int factor){
  int retVal = 0;

    for(unsigned int i=0; i <= dim-1;i++){
      retVal = retVal+(data[i]*factor);
      factor = factor/10;
   }  
   return retVal;
}
\end{lstlisting}

\subsubsection{Matlab Code}
Die entsprechenden Befehle können über ein Matlab-Skript (s. 
\ref{lst:matlabScript}) an das Board OpenCM gesendet werden. 

\label{lst:matlabScript}
\begin{lstlisting}[language=java, caption=Matlab-Skript]
function ComDeltaRobot()    
    pTime = 0.8;
    
    PortID = serial( 'COM3' );
    set( PortID,'BaudRate',9600, 'StopBits', 2);
    
    fopen( PortID );
    
    fprintf(PortID,'1;30;55;');
    pause(pTime);
    
    fprintf(PortID,'1;30;0;');
    pause(pTime);

    fprintf(PortID,'1;30;55;');
    pause(pTime);
    
    fprintf(PortID,'1;30;0;');
    pause(pTime);
    
    OutPort = instrfind('Port','COM3');
    NewObjs = instrfind;
    fclose(NewObjs);   
end
\end{lstlisting}
