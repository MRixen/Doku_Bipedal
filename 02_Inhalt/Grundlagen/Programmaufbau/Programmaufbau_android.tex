\subsection{Android}
\label{sec:ChapterAndroid}
Das System besteht aus mehreren Klassen von denen die wichtigsten folgend 
Erläuterung finden. Die Klassen sind der Reihenfolge nach den Tasks der 
Robotersteuerung zuzuordnen. D.h. die Klasse \textit{Receiver} ist für die 
grundsätzliche Kommunikation zuständig - entsprechend des Tasks 
\textit{ServerComm}, etc.

\subsubsection{Klasse Receiver}
Die Klasse \textit{Receiver} ist für das Senden und Empfangen von Daten 
zuständig. Die grundsätzliche Kommunikation erfolgt in der Klasse 
\textit{NetClient}, 
welche innerhalb des Konstruktors (s. \ref{appCode:klasseReceiver} Zeile 1 - 5) 
als Objekt erzeugt wird. Die Übergabeparameter bestehen aus der IP-Adresse und 
dem Port.

Die Klasse Receiver implementiert ein Runnable, sodass innerhalb der Routine 
run() die Kommunikation mit einem Client abläuft. In Zeile 9 erfolgt der 
Verbindungsaufbau. Ist dieser erfolgreich, so beginnt das Empfangen von Daten 
innerhalb einer While-Schleife (Zeile 11 - 20). Die empfangenen Daten werden 
durch Events an alle Teilnehmer (Klasse Events, MachineData und CycleTime) 
gesendet innerhalb derer die Nachricht entsprechend des Inhalts von 
msgType extrahiert wird.\\
Ist die Verbindung nicht erfolgreich, so erfolgt die Ausgabe einer Meldung auf 
dem Display und die App wird nach einer Zeit x beendet (s. Zeile 22 - 32).

Mit der Routine \textit{registerListener()} erfolgt das Registrieren auf den 
Event-Generator. Dieses Interface beinhaltet die Methode onEvent(), die zwei 
Übergabeparameter aufweist, um den Nachrichtentyp (msgType z.B. :e: für Event) 
und die Nachricht selbst zu übertragen.

\label{appCode:klasseReceiver}
\begin{lstlisting}[language=java, caption=Klasse Receiver]
    public Receiver(Context context, String ip, String port, Activity activity){
        (...)
        if (nc == null) nc = new NetClient(this.ip, 
        Integer.parseInt(this.port));
    }
	(...)

public void run() {
	if (nc.connectWithServer()) {
	    (...)
        while (isRunning) {
	        data = nc.receiveDataFromServer();
			if ((data[0] != null) && (data[1] != null)) {
	            (...)
                 for (EventListener eventListener : listeners)
	                 eventListener.onEvent(data[0], data[1]);
                     (...)
            }
        }
    }
	nc.disConnectWithServer();
    else {
		(...)
        Toast.makeText(context, R.string.connectionError, 
        Toast.LENGTH_LONG).show();
        Timer t = new Timer();
        t.schedule(new TimerTask() {
        public void run() {
		    activity.finish();
		}
	}, maxActivityShowTime);
}

public void registerListener(EventListener listener) {
	this.listeners.add(listener);
}

public interface EventListener {
    void onEvent(String data1, String data2);
	(...)
}
\end{lstlisting}


\subsubsection{Klasse Events}
Die Klasse \textit{Events} definiert einen Tab und empfängt/verarbeitet die von 
der Robotersteuerung gesendeten Events (Information, Warnung, Fehler). Die 
Klasse registriert sich (wie in allen anderen Tab-Klassen auch) in ihrer 
Methode \textit{onCreate()} auf den Event-Generator, bzw. das Interface 
\textit{EventListener} (s. \ref{appCode:klasseEvents} Zeile 7 - 8).\\
Sobald die Robotersteuerung einen String sendet, erfolgt der Aufruf von 
\textit{onEvent()} und einhergehend der Aufruf von \textit{showEvent()} (s. 
Zeile 12 - 14 und 16 - 22). Daraufhin wird geprüft, ob der Nachrichtentyp 
(msgType) das 
Zeichen \textit{e} beinhaltet (s. Zeile 17). Ist dies der Fall so wird der 
String aufgeteilt und dem Objekt \textit{XMLParsing()} übergeben (s. Zeile 20).

Die Klasse \textit{XMLParsing} erzeugt das String-Array 
$\textit{eventDescription[]}$, das den gesamten Meldungstext 
beinhaltet. D.h. anhand der Parameter Fehlerdomäne und Fehlernummer erfolgt die 
Auswahl der xml-Datei (Zeile 30 - 40 verkürzt dargestellt) in welcher der 
Beschreibungstext für die Meldung hinterlegt ist.\\
Das Array $\textit{eventMessages[]}$ enthält an der Position 1 die 
Fehlerdomäne, an Position 2 die Fehlernummer und an Position 3 den Fehlertyp. 
Alle weiteren Positionen beinhalten die Textparameter.

Das Parsen der entsprechenden xml-Datei erfolgt in Zeile 45 - 78 mit einer 
Switch-Case Struktur. In dem ersten Case (Zeile 48 - 56) wird (innerhalb von 
if-Abfragen) der Ort eines xml-Eintrages gesucht und die hinterlegten Strings 
in das Array \textit{eventDescription$[$$]$} kopiert (die Cases sind verkürzt 
dargestellt). Im zweiten Case (Zeile 70 - 72) wird das Ende des Dokuments 
detektiert.\\
Ist das Parsen beendet erfolgt der Methoden-Aufruf \textit{onPostExecute()} 
(Zeile 80 - 85) und der Meldungstext wird sofort als Dialog mit Vibration 
dargestellt (sofern die Motoren ausgeschaltet sind, bzw. eventMessages$[$0$]$ 
== 0 ist) oder in eine Liste geschrieben.

Die xml-Dateien wurden von der Robotersteuerung extrahiert und in die 
App-Struktur eingefügt.

\label{appCode:klasseEvents}
\begin{lstlisting}[language=java, caption=Klasse Events]
protected void onCreate(Bundle savedInstanceState) {
(...)
	Bundle bundle = getIntent().getExtras();
    BaseData baseData = (BaseData) bundle.getSerializable("baseData");
    if (receiver == null) {
	    (...)
        receiver = baseData.getReceiver();
        receiver.registerListener(this);
        (...)
}

public void onEvent(String msg, String msgType) {
	showEvent(msgType, msg, true);
}

private void showEvent(String msgType, String eventMessage, boolean saveEvent) {
	if (msgType.equals("e")){
	    if (saveEvent) listViewEntryData[listCounter] = eventMessage;
        String[] tempMessage = eventMessage.split(":");
        new XMLParsing(saveEvent, this).execute(tempMessage);
    }
}

    private class XMLParsing extends AsyncTask<String, Void, String[]> {
(...)
        protected String[] doInBackground(String... eventMessages) {
            this.eventMessages = eventMessages;
            String[] eventDescription = new String[] {"", "", "", "", "", ""};
(...)
                switch (Integer.parseInt(eventMessages[1])){
                    case 1:
                        if (Integer.parseInt(eventMessages[2]) <= 175) filename 
                        = "elog/"+"opr_"+"elogtext_"+1+".xml";
                        if ((Integer.parseInt(eventMessages[2]) > 175) && 
                        (Integer.parseInt(eventMessages[2]) <= 1231)) filename 
                        = "elog/"+"opr_"+"elogtext_"+2+".xml";
                        (...)
                        break;
                    case 2:
                        (...)
                }
                InputStream in_s = getAssets().open(filename);
                (...)

                while ((event != XmlPullParser.END_DOCUMENT) && !messageReadOk)
                {
                    switch (event){
                        case XmlPullParser.START_TAG:
                            name = xmlParser.getName();
                            if(name.equals("Message") && !messageEntryFound) {
                                String messageNumber = 
                                xmlParser.getAttributeValue(null,"number");
                                if (messageNumber.equals(eventMessages[2])){
                                    messageEntryFound = true;
                                }
                            }
                            if(messageEntryFound && name.equals("Title")) 
                            eventDescription[0] = xmlParser.nextText();
                            if(messageEntryFound && name.equals("Description")){
                                eventDescription[1] = xmlParser.nextText();
                                eventDescription[1] = 
                                String.format(eventDescription[1], 
                                eventMessages[4], eventMessages[5], 
                                eventMessages[6], eventMessages[7], 
                                eventMessages[8]);
                            }
(...)
                            break;

                        case XmlPullParser.END_TAG:
(...)
                    }
                    if (!messageReadOk){
(...)
        }
        eventDescription[5] = eventMessages[3];
                    return eventDescription;
                    }

        protected void onPostExecute(String[] result) {
            if (addEvents) setEventList(result[0]);
            if ((eventMessages[0].equals("0")) || !addEvents){
                customEventDialog.showDialog(result);
                if (addEvents) vibrator.vibrate(800);
            }
        }
    }
\end{lstlisting}


\subsubsection{Klasse MachineData}
Die Klasse \textit{MachineData} definiert einen Tab und zeigt tabbelarisch die 
Maschinendaten auf. Die Klasse registriert sich (wie in allen anderen 
Tab-Klassen auch) in ihrer 
Methode \textit{onCreate()} auf den Event-Generator, bzw. das Interface 
\textit{EventListener} (s. \ref{appCode:klasseMachineData} Zeile 7 - 8).\\
Sobald die Robotersteuerung einen String sendet, erfolgt der Aufruf von 
\textit{onEvent()} und einhergehend der Aufruf von \textit{showEvent()} (s. 
Zeile 12 - 14 und 16 - 27). Daraufhin wird geprüft, ob der Nachrichtentyp 
(msgType) eine Zahl (0, 1, 2, n) enthält (s. Zeile 18 - 21). Ist dies der 
Fall so wird der 
String der entsprechenden Zeile in der Tabelle übergeben (s. Zeile 24 - 25).

\label{appCode:klasseMachineData}
\begin{lstlisting}[language=java, caption=Klasse MachineData]
    protected void onCreate(Bundle savedInstanceState) {
(...)
        Bundle bundle = getIntent().getExtras();
        BaseData baseData = (BaseData) bundle.getSerializable("baseData");
        if (receiver == null) {
            (...)
                receiver = baseData.getReceiver();
                receiver.registerListener(this);
            (...)
    }
    
        public void onEvent(String data1, String data2) {
            showEvent(data2, data1);
        }
        
        private void showEvent(String msgType, String eventMessage) {
            int number = -1;
                    try {
                        number = Integer.parseInt(msgType);
                    } catch (NumberFormatException e) {
                    }
            if (number != -1) {
            (...)
                machineData[Integer.parseInt(msgType)] = eventMessage;
                textViews[Integer.parseInt(msgType)].setText(eventMessage);
            } 
            (...)
        }

\end{lstlisting}


\subsubsection{Klasse CycleTime}
Die Klasse \textit{CycleTime} definiert einen Tab und stellt den 
aktuellen, sowie den Durchschnitts-Wert der Zykluszeit dar. Die Klasse 
registriert sich (wie in allen anderen 
Tab-Klassen auch) in ihrer 
Methode \textit{onCreate()} auf den Event-Generator, bzw. das Interface 
\textit{EventListener} (s. \ref{appCode:klasseCycleTime} Zeile 7 - 8).\\
Sobald die Robotersteuerung einen String sendet, erfolgt der Aufruf von 
\textit{onEvent()} und einhergehend der Aufruf von \textit{showEvent()} (s. 
Zeile 12 - 14 und 16 - 25). Daraufhin wird geprüft, ob der Nachrichtentyp 
(msgType) das Zeichen \textit{c1} oder \textit{c2} enthält (s. Zeile 17 - 24). 
Ist dies der Fall so wird der 
String dem entsprechenden Textfeld übergeben (s. Zeile 19 und 23).

\label{appCode:klasseCycleTime}
\begin{lstlisting}[language=java, caption=Klasse CycleTime]
    protected void onCreate(Bundle savedInstanceState) {
(...)
        Bundle bundle = getIntent().getExtras();
        BaseData baseData = (BaseData) bundle.getSerializable("baseData");
        if (receiver == null) {
            (...)
                receiver = baseData.getReceiver();
                receiver.registerListener(this);
            (...)
    }
    
        public void onEvent(String msg, String msgType) {
            showMessage(msgType, msg);
        }
        
            private void showMessage(String msgType, String msg) {
                if (msgType.equals("c1")) {
                    String msgTemp = msg.replace(".", ",");
                    cycleTimeViewer_actual.setText(msgTemp);
                }
                if (msgType.equals("c2")) {
                    String msgTemp = msg.replace(".", ",");
                    cycleTimeViewer_mean.setText(msgTemp);
                }
            }
\end{lstlisting}


\subsubsection{Klasse Logging}
Die Klasse \textit{Logging} definiert einen Tab und stellt die 
Logging-Nachrichten dar, die der User ansonsten auf dem FlexPendant unter 
Logging 
sehen würde. Die Klasse registriert sich (wie in allen anderen 
Tab-Klassen auch) in ihrer 
Methode \textit{onCreate()} auf den Event-Generator, bzw. das Interface 
\textit{EventListener} (s. \ref{appCode:klasseLogging} Zeile 7 - 8).\\
Sobald die Robotersteuerung einen String sendet, erfolgt der Aufruf von 
\textit{onEvent()} und einhergehend der Aufruf von \textit{showEvent()} (s. 
Zeile 12 - 14 und 16 - 28). Daraufhin wird geprüft, ob der Nachrichtentyp 
(msgType) das Zeichen \textit{l} enthält (s. Zeile 17). 
Ist dies der Fall so wird der 
String einer Liste hinzugefügt (s. Zeile 22 - 23) mit dem aktuellsten Eintrag 
an oberster Stelle. Die maximale Länge der Liste beläuft sich aktuell auf 10. 
Ist diese überschritten, so wird der letzte Listeneintrag gelöscht.

\label{appCode:klasseLogging}
\begin{lstlisting}[language=java, caption=Klasse Logging]
    protected void onCreate(Bundle savedInstanceState) {
(...)
        Bundle bundle = getIntent().getExtras();
        BaseData baseData = (BaseData) bundle.getSerializable("baseData");
        if (receiver == null) {
            (...)
                receiver = baseData.getReceiver();
                receiver.registerListener(this);
            (...)
    }
    
        public void onEvent(String msg, String msgType) {
            showMessage(msgType, msg);
        }
        
    private void showMessage(String msgType, String msg) {
        if (msgType.equals("l")) {
            int MAX_LOG_AMOUNT = 10;
            if (loggingList.size() >= MAX_LOG_AMOUNT) {
                loggingList.remove(loggingList.size() - 1);
            }
            loggingList.add(0, msg);
            arrayAdapter.notifyDataSetChanged();
            int MAX_LOG_COUNTER = 100;
            if (logCounter <= MAX_LOG_COUNTER - 1) logCounter += 1;
            else logCounter = 0;
        }
    }
\end{lstlisting}


Alle weiteren Klassen, die bei dem Zusammenspiel zwischen der Robotersteuerung 
und der App beteiligt sind finden im Folgenden Erläuterung. Dabei wird kurz die 
Funktion erläutert.

\textbf{MainActivity}\\
In dieser Klasse werden die Tabs generiert und dargestellt, sowie der 
Barcode-Reader gestartet und auf dessen Ergebnis gewartet. Ist dieses i.O. so 
erfolgt das Starten des Receiver-Threads.

\textbf{BaseData}\\
Diese Klasse ist für die Datenablage gedacht, die von verschiedenen Klassen  
benötigt werden.

\textbf{NetClient}\\
Die Klasse \textit{NetClient} ist für die grundsätzliche Kommunikation 
zuständig. D.h. Verbindungsauf- und abbau, sowie Senden und Empfangen von 
Daten. In der Methode \textit{receiveDataFromServer()} wird bspw. jedes 
Character eingelesen.

\textbf{CustomDialogEvent}\\
Diese Klasse erzeugt den Dialog, welcher alle Komponenten für die Darstellung 
des Meldungstextes enthält (Information, Warnung, Fehler).

Es existieren noch weitere Dialog-Klassen, welche für bestimmte Anwendungsfälle 
konzipiert sind. So wird bspw. mit der Klasse \textit{CustomDialogAbout} die 
About-Information dargestellt. Alle weiteren Dialog-Klassen werden aktuell 
nicht genutzt.

\textbf{BarCodeReading}\\
Diese Klasse erzeugt den QR-Code Reader, validiert den Inhalt des Codes und 
gibt eine Rückmeldung an die Klasse \textit{MainActivity} zurück.

\textbf{CycleTimeDrawThread}\\
Um die Zykluszeit grafisch darstellen zu können ist diese Klasse hilfreich. 
Darin enthalten sind Methoden, mit denen einen grafische Ausgabe (Diagramm) 
erzeugt werden kann. Aktuell wird die Klasse nicht genutzt.

