%%% Einheiten korrekt setzen (2 Varianten)

% -------------------------------------------------
% Variante 1: Alles manuell einstellen
% Einheiten korrekt setzen
%\usepackage{siunitx}
% Hier: Ergänzung zu siunitx (\sfrac)
%\usepackage{xfrac}
% Konfiguration von siunitx
%\sisetup{
  %fraction-function = \sfrac,
  %per-mode          = fraction,
  %decimalsymbol		= comma, % Komma statt Punkt
  %exponent-product  = \cdot, % 2x10^2 kg vs. 2.10^2 kg
  %list-final-separator = { und },
  %range-phrase = { bis },
  %separate-uncertainty = true,
  %inter-unit-separator={}\cdot{}, % m s vs. m.s
%}
% -------------------------------------------------

%% -------------------------------------------------
%% Variante 2: Über Zusatz-Pakete und Paketoptionen konfigurieren
%% Hier: Damit siunitx weiß, welche Sprache eingestellt ist
%% ngerman kann auch als Klassenoption übergeben werden
\usepackage[ngerman]{translator}
%% Einheiten korrekt setzen
\usepackage{siunitx}
%% Hier: Ergänzung zu siunitx (\sfrac)
\usepackage{xfrac}
%% Konfiguration von siunitx
\sisetup{
  locale=DE, % Komma statt Punkt \SI{1.3}{m} -> 1,3 m
  fraction-function = \sfrac,
  per-mode          = fraction,
  %exponent-product  = \cdot, % 2x10^2 kg vs. 2.10^2 kg
  separate-uncertainty = true,
  %inter-unit-separator={}\cdot{}, % m s vs. m.s
}
%% -------------------------------------------------

% Problem Leerzeichen nach Befehlen lösen
\usepackage{xspace}

\usepackage[onehalfspacing]{setspace}

