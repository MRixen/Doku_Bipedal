\documentclass[12pt,a4paper,bibtotoc, liststotoc, headsepline]{scrreprt}

%\documentclass[
%a4paper,
%12pt,
%oneside,
%headings=big,
%chapterprefix,
%headsepline
%]{scrbook}
%\usepackage[T1]{fontenc} % Ausgabe-Encoding
%\usepackage[utf8]{inputenc}% Eingabe-Encoding
%\usepackage[ngerman]{babel}% deutsch
%\usepackage{scrpage2} % Kopf und Fuß
%
%\pagestyle{scrheadings}
%\setkomafont{pageheadfoot}{\normalfont\bfseries}
%\renewcommand*\chapterpagestyle{scrheadings}
%\renewcommand*\sectionmark[1]{\markright{\thesection\ #1}} 
%
%% Kopf
%\ihead{} % innen oder links
%\chead{}
%\ohead{\headmark} % außen oder rechts
%
%% Fuß
%\ifoot{} % innen oder lnks
%\cfoot{}
%\ofoot{\pagemark} % außen oder rechts



%\documentclass{scrreprt}
%\usepackage{blindtext}
%\usepackage[automark]{scrlayer-scrpage}
%\pagestyle{scrheadings}
%\ifoot[\TeX nische Universit\"at]{\TeX nische Universit\"at}
%%\chead[\headmark]{\headmark}
%\begin{document}
%\blinddocument
%
%\pagestyle{plain}
%%Von nun an nur noch im style plain
%\blinddocument
%
%\pagestyle{empty}
%\chapter{Und nun empty}
%%Die erste Seite wird natürlich noch im Stil plain gesetzt.
%%Also legen wir direkt nach chapter den Seitenstil lokal fest. 
%\thispagestyle{empty}
%\blindtext[6]
%\end{document}
%\documentclass[ 
%12pt, 
%a4paper, 
%headinclude, 
%footinclude, 
%plainfootsepline]{scrreprt} 
%\usepackage{scrpage2} 
%
%\pagestyle{scrheadings} 
%\clearscrheadfoot 
%\automark{chapter} 
%   \ihead{\headmark}   
%   \cfoot[-{ }\pagemark{ }-]{-{ }\pagemark{ }-} 
%%Bei der Einstellung der Seitenstile bietet KOMA-Script direkt die Möglichkeit, alles für beide "Haupt"seitenstile einzustellen 
%%\Position[plain-Seitenstil]{scrheadings-Seitenstil} 
%\setheadsepline{0.5pt} 
%\setfootsepline{0.5pt} 